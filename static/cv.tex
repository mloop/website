% Jason R. Blevins - Curriculum Vitae
%
% Copyright (C) 2004-2011 Jason R. Blevins
%
% You may use use this document as a template to create your own CV
% and you may redistribute the source code freely.  No attribution is
% required in any resulting documents.  I do ask that you please leave
% this notice and the above URL in the source code if you choose to
% redistribute this file.

\documentclass[10pt,letterpaper]{article}

\usepackage[colorlinks=true,linkcolor=black,anchorcolor=black,citecolor=black,filecolor=black,menucolor=black,runcolor=black,urlcolor=black]{hyperref}
\usepackage{geometry}
\usepackage{etaremune}

% Fonts
%\usepackage[T1]{fontenc}
%\usepackage[urw-garamond]{mathdesign}

% Set your name here
\def\name{Matthew Shane Loop, PhD FAHA}

\geometry{
  body={6.5in, 9.0in},
  left=1.0in,
  top=1.0in
}

% Customize page headers
\pagestyle{myheadings}
\markright{\name}
\thispagestyle{empty}

% Custom section fonts
%\sectionfont{\rmfamily\mdseries\Large}
%\subsectionfont{\rmfamily\mdseries\itshape\large}

% Other possible font commands include:
% \ttfamily for teletype,
% \sffamily for sans serif,
% \bfseries for bold,
% \scshape for small caps,
% \normalsize, \large, \Large, \LARGE sizes.

% Don't indent paragraphs.
\setlength\parindent{0em}

% Make lists without bullets and compact spacing
\renewenvironment{itemize}{
  \begin{list}{}{
    \setlength{\leftmargin}{1.5em}
    \setlength{\itemsep}{0.25em}
    \setlength{\parskip}{0pt}
    \setlength{\parsep}{0.25em}
  }
}{
  \end{list}
}

\begin{document}

% Place name at left
{\Huge \name}

% Alternatively, print name centered and bold:
%\centerline{\huge \bf \name}

\bigskip

\begin{minipage}[t]{0.6\textwidth}
  Division of Pharmacotherapy and Experimental Therapeutics \\
  UNC Eshelman School of Pharmacy\\
  University of North Carolina at Chapel Hill\\
  301 Pharmacy Lane\\
  3208 Kerr Hall, CB\#7569 \\
  Chapel Hill, NC 27599
\end{minipage}
\begin{minipage}[t]{0.5\textwidth}
  Telephone: (919) 962-5339\\
  Email: \href{mailto:matthew\_loop@unc.edu}{matthew\_loop@unc.edu} \\
\end{minipage}

\bigskip

\section*{Education}

\begin{itemize}
  \item Postdoctoral Fellowship, Department of Epidemiology, UAB Vascular Biology and Hypertension Program, UAB, April 2015 - August 2016
  \begin{itemize}
    \item Advisors: Emily B. Levitan, ScD (primary) and Paul M. Muntner, PhD (secondary)
  \end{itemize}
	\item PhD in Biostatistics, University of Alabama at Birmingham (UAB), May 2015
		\begin{itemize}
			\item Advisor: Leslie A. McClure, PhD
		\end{itemize}
	\item M.S. in Biostatistics, UAB, May 2012
		\begin{itemize}
			\item Advisor: Leslie A. McClure, PhD
		\end{itemize}
  	\item B.S. Biology, University of Alabama, \textit{cum laude}, May 2010
\end{itemize}

\section*{Positions}
\begin{itemize}
    \item Assistant Professor, Division of Pharmacotherapy and Experimental Therapeutics, UNC Eshelman School of Pharmacy, University of North Carolina at Chapel Hill, August 2020 - 
		\item Assistant Professor, Department of Biostatistics, Gillings School of Global Public Health, University of North Carolina at Chapel Hill, September 2016 - July 2020
\end{itemize}

\section*{Peer-Reviewed Papers}

$^*$ Indicates student I advised on the project.

\begin{etaremune}

\item Tsao CW, Aday AW, Almarzooq ZI, Alonso A, Beaton AZ, Bittencourt MS, Boehme AK, Buxton AE, Carson AP, Commodore-Mensah Y, Elkind MSV, Evenson KR, Eze-Nliam C, Ferguson JF, Generoso G, Ho JE, Kalani R, Khan SS, Kissela BM, Knutson KL, Levine DA, Lewis TT, Liu J, \textbf{Loop MS}, Ma J, Mussolino ME, Navaneethan SD, Perak AM, Poudel R, Rezk-Hanna M, Roth GA, Schroeder EB, Shah SH, Thacker EL, VanWagner LB, Virani SS, Voecks JH, Wang N-Y, Yaffe K, \&  Martin SS; on behalf of the American Heart Association Council on Epidemiology and Prevention Statistics Committee and Stroke Statistics Subcommittee (2022). Heart disease and stroke statistics—2022 update: a report from the American Heart Association [published online ahead of print Wednesday, January 26, 2022]. Circulation. doi: 10.1161/CIR.0000000000001052.

\item Kucharska-Newton AM, \textbf{Loop MS}, Bullo M, Moore C, Haas SW, Wagenknecht L, Whitsel E, \& Heiss G (2021). Use of troponins in the classification of myocardial infarction from electronic health records. The Atherosclerosis Risk in Communities (ARIC) Study. \emph{International Journal of Cardiology}. {In press}.

\item Mulrenin IR, Garcia JE, Fashe MM, \textbf{Loop MS}, Daubert MA, Urrutia RP, \& Lee CR (2021). The impact of pregnancy on antihypertensive drug metabolism and pharmacokinetics: current status and future directions. \emph{Expert Opinion on Drug Metabolism and Toxicology}. \emph{In press}.

\item Lemas DJ, \textbf{Loop MS}, Duong M, Schleffer AB, Collins C, Bowden JA, Du X, Patel KU, Ciesielski AL, Ridge Z, Wagner J, Subedi B, \& Delcher C (2021). Estimating drug consumption during a college sporting event from wastewater using liquid chromatography mass spectrometry. \emph{Science of the Total Environment} 764:143963. doi: 10.1016/j.scitotenv.2020.143963.

\item Goff DC Jr, Khan SS, Arnett DK, Carnethon MR, Labarthe DR, \textbf{Loop MS}, Luepker RV, McConnell MV, Mensah GA, Mujahid M, O'Flaherty M, Prabhakaran D, Roger VL, Rosamond WD, Sidney S, Wei GS, Wrigh JS, \& Lloyd-Jones DM (2021). Bending the curve in cardiovascular disease mortality: Bethesda $+40$ and beyond. \emph{Circulation} 143(8): 837-851.

\item Virani SS, Alonso A, Aparicio HJ, Benjamin EJ, Bittencourt MS, Callaway CW, Carson AP, Chamberlain AM, Cheng S, Delling FN, Elkind MSV, Evenson KR, Ferguson JF, Gupta DK, Khan SS, Kissela BM, Knutson KL, Lee CD, Lewis TT, Liu J, \textbf{Loop MS}, Lutsey PL, Ma J, Mackey J, Martin SS, Matchar DB, Mussolino ME, Navaneethan SD, Perak AM, Roth GA, Samad Z, Satou GM, Schroeder EB, Shah SH, Shay CM, Stokes A, VanWagner LB, Wang N-Y, Tsao CW; on behalf of the American Heart Association Council on Epidemiology and Prevention Statistics Committee and Stroke Statistics Subcommittee (2021). Heart disease and stroke statistics—2021 update: a report from the American Heart Association. \emph{Circulation}. 143:e00–e00. doi: 10.1161/CIR.0000000000000950.

\item Kohn JN, \textbf{Loop MS}, Kim-Chang JJ, Garvie PA, Sleasman JW, Fischer B, Rendina HJ, Woods SP, Nichols SL, Hong S (2021). Trajectories of depressive symptoms, neurocognitive function and viral suppression with antiretroviral therapy among youth with HIV over 36 months. \emph{JAIDS} 87(2): 851-859.

\item DeMonte JB, Neilan AM, \textbf{Loop MS}, Ciaranello AL, \& Hudgens MG (2021). Rapid report on estimating incidence from cross-sectional data. \emph{Annals of Epidemiology} 53: 106 - 108. e1.

\item \href{https://bmcpublichealth.biomedcentral.com/articles/10.1186/s12889-020-09039-z}{Bell GJ$^*$, \textbf{Loop MS}, Mvalo T, Juliano JJ, Mofolo I, Kamthunzi P, Tegha G, Lievens M, Bailey J, Emch M, \& Hoffman I (2020). Environmental modifiers of RTS,S/AS01 malaria vaccine efficacy in Lilongwe, Malawi. \emph{BMC Public Health}. 20:910. doi: 10.1186/s12889-020-09039-z.}

\item \href{https://www.ahajournals.org/doi/pdf/10.1161/JAHA.120.016122}{Burroughs Pe\~{n}a MS, Uwamungu JC, Bulka CM, Swett K, Perreira KM, Kansal MM, \textbf{Loop MS}, Hurwitz BE, Daviglus M, Rodriguez CJ (2020). Occupational Exposures and Cardiac Structure and Function: ECHO-SOL (Echocardiographic Study of Latinos). \emph{Journal of the American Heart Association} 9:e016122. doi: 10.1161/JAHA.120.016122.}

\item \href{https://journals.plos.org/plosone/article?id=10.1371/journal.pone.0233161}{\textbf{Loop MS}, Van Dyke MK, Chen L, Brown TM, Duran RW, Safford MK, Levitan EB (2020). Evidence-based beta blocker use associated with lower heart failure readmission and mortality, but not all-cause readmission, among Medicare beneficiaries hospitalized for heart failure with reduced ejection fraction. \emph{PLOS ONE} 15(7):e0233161. doi:10.1371/journal.pone.0233161.}

\item \href{https://www.sciencedirect.com/science/article/pii/S0264410X20305119}{Bell GJ$^*$, \textbf{Loop MS}, Topazian H, Hudgens M, Mvalo T, Juliano JJ, Kamthunzi P, Tegha G, Mofolo I, Hoffman I, Bailey JA, Emch M (2020). Case Reduction and Cost-Effectiveness of the RTS,S Malaria Vaccine Alongside Bed Nets in Lilongwe, Malawi. \emph{Vaccine} 38: 4079 - 4087.}

\item \href{https://jasn.asnjournals.org/content/jnephrol/31/6/1315.full.pdf}{Ricardo AC, \textbf{Loop MS}, Gonzalez II F, Lora CM, Chen J, Franceschini N, Kramer HJ, Toth-Manikowski S, Talavera GA, Daviglus ML, Lash JP. Incident Chronic Kidney Disease Risk Among U.S. Hispanics/Latinos: The Hispanic Community Health Study/Study of Latinos (HCHS/SOL). \emph{J Am Soc Nephrol.} 31:1315-1324.}

\item \href{https://www.ahajournals.org/doi/abs/10.1161/CIR.0000000000000757}{Virani SS, Alonso A, Benjamin EJ, Bittencourt MS, Callaway CW, Carson AP, Chamberlain AM, Chang AR, Cheng S, Delling FN, Djousse L, Elkind MSV, Ferguson JF, Fornage M, Khan SS, Kissela BM, Knutson KL, Kwan TW, Lackland DT, Lewis TT, Lichtman JH, Longenecker CT, \textbf{Loop MS}, Lutsey PL, Mar-tin SS, Matsushita K, Moran AE, Mussolino ME, Perak AM, Rosamond WD, Roth GA, Sampson UKA, Satou GM, Schroeder EB, Shah SH, Shay CM, Spartano NL, Stokes A, Tirschwell DL, VanWagner LB, Tsao CW; on behalf of the American Heart Association Council on Epidemiology and Prevention Statistics Committee and Stroke Statistics Subcommittee. Heart disease and stroke statistics— 2020 update: a report from the American Heart Association. \emph{Circulation}. 2020;141:e1–e458. doi: 10.1161/CIR.0000000000000757.}

\item \href{https://www.researchprotocols.org/2020/8/e16384/}{Hill BJ, Motley DN, Rosentel K, VandeVusse A, Garofalo R, Schneider JA, Kuhns LM, Kipke MD, Reisner S, Rupp BM, Sanchez M, McCumber M, Renshaw L, \textbf{Loop MS} (2020). An employment intervention program (Work2Prevent) for young men who have sex with men and transgender youth of color (Phase 1): Protocol for determining essential intervention components using qualitative interviews and focus groups. \emph{JMIR Research Protocols} 9(8):e16384. doi: 10.2196/16384.}

\item \href{https://www.researchprotocols.org/2020/8/e16401/}{Hill BJ, Motley DN, Rosentel K, VandeVusse A, Garofalo R, Schneider JA, Kuhns L, Kipke M, Reisner S, Rupp BM, Sanchez M, McCumber M, Renshaw L, West Goolsby R, \textbf{Loop MS}. Work2Prevent: Protocol for employment as HIV prevention for young men who have sex with men (YMSM) and transgender youth of color – Phase 2 (ATN 151). \emph{JMIR Research Protocols} 9(8):e16401. doi: 10.2196/16401.}


\item \href{https://journals.lww.com/aidsonline/Fulltext/2019/12010/Higher_soluble_CD14_levels_are_associated_with.9.aspx}{Kim-Chang JJ, Donovan K$^*$, \textbf{Loop MS}, Hong S, Fischer B, Venturi G, Garvie PA, Kohn J, Rendina HJ, Woods SP, Goodenow MM, Nichols SL, \& Sleasman JW (2019). Higher soluble CD14 levels are associated with lower visuospatial memory performance in Youth with HIV (YWH). \emph{AIDS} 33(15): 2363 - 2374. doi: 10.1097/QAD.0000000000002371}



\item \href{https://www.ahajournals.org/doi/abs/10.1161/CIR.0000000000000659}{Benjamin EJ, Muntner P, Alonso A, Bittencourt MS, Callaway CW, Carson AP, Chamberlain AM, Chang AR, Cheng S, Das SR, Delling FN, Djousse L, Elkind MSV, Ferguson JF, Fornage M, Jordan LC, Khan SS, Kissela BM, Knutson KL, Kwan TW, Lackland DT, Lewis TT, Lichtman JH, Longenecker CT, \textbf{Loop MS}, Lutsey PL, Martin SS, Matsushita K, Moran AE, Mussolino ME, O’Flaherty M, Pandey A, Perak AM, Rosamond WD, Roth GA, Sampson UKA, Satou GM, Schroeder EB, Shah SH, Spartano NL, Stokes A, Tirschwell DL, Tsao CW, Turakhia MP, VanWagner LB, Wilkins JT, Wong SS, \& Virani SS; on behalf of the American Heart Association Council on Epidemiology and Prevention Statistics Committee and Stroke Statistics Subcommittee. Heart disease and stroke statistics - 2019 update: a report from the American Heart Association [published online ahead of print January 31, 2019]. \emph{Circulation}. doi: 10.1161/CIR.0000000000000659.}

\item \href{https://journals.plos.org/plosone/article?id=10.1371/journal.pone.0207652}{Estrella ML, Rosenberg NI, Durazo-Arvizu RA, Gonzalez HM, \textbf{Loop MS}, Singer RH, Lash JP, Castañeda SF, Perreira KM, Eldeirawi K, Daviglus ML (2018). The association of employment status with ideal cardiovascular health factors and behaviors among Hispanic/Latino adults: Findings from the Hispanic Community Health Study/Study of Latinos (HCHS/SOL). \emph{PLOS ONE} 27;13(11):e0207652. doi: 10.1371/journal.pone.0207652. eCollection 2018. PubMed PMID: 30481192.}

\item \href{https://www.sciencedirect.com/science/article/pii/S0168822718308866}{Casagrande SS, Menke A, Aviles-Santa L, Gallo LC, Daviglus M, Talavera GA, Castaneda SF, Perreira K, \textbf{Loop MS}, Tarraf W, Gonzalez HM, \& Cowie CC (2018). Factors associated with undiagnosed diabetes among adults with diabetes: Results from the Hispanic Community Health Study/Study of Latinos (HCHS/SOL). \emph{Diabetes Research and Clinical Practice} 146: 258-266.}

\item  \href{https://www.sciencedirect.com/science/article/abs/pii/S1071916418311072}{\textbf{Loop MS}, Van Dyke MK, Chen L, Safford MM, Kilgore ML, Brown TM, Durant RW, Levitan EB (2018). Low utilization of beta blockers among Medicare beneficiaries hospitalized for heart failure with reduced ejection fraction. \emph{Journal of Cardiac
Failure}. doi:10.1016/j.cardfail.2018.10.00}

\item \href{https://www.ahajournals.org/doi/full/10.1161/JAHA.117.007785}{Goyal P, \textbf{Loop MS}, Chen L, Brown TM, Durant RW, Safford MM, \& Levitan EB (2018).  Causes and Temporal Patterns of 30-day Readmission Among Older Adults Hospitalized with Heart Failure. \emph{Journal of the American Heart Association} 7 (9), e007785.}

\item \href{https://www.sciencedirect.com/science/article/pii/S0002870317303630}{\textbf{Loop MS}, McClure LA, Levitan EB, Al-Hamdan MZ, Crosson WL, \& Safford MM (2017). Fine particulate matter and incident coronary heart disease in the REGARDS cohort. \emph{American Heart Journal} 197: 94-102.}

\item \href{https://www.ingentaconnect.com/content/wk/mbp/2018/00000023/00000002/art00007}{Bromfield SG, Booth, III, JN, \textbf{Loop MS}, Schwartz JE, Seals SR, Thomas SJ, Ming Y-I, Ogedegbe G, Shimbo D, \& Muntner P (2017). Evaluating different criteria for defining a complete ambulatory blood pressure monitoring recording: Data from the Jackson Heart Study. \emph{Blood Pressure Monitoring} 23 (2): 103-111.}

\item \href{https://link.springer.com/article/10.1007/s10557-017-6764-8}{Levitan EB, Van Dyke MK, \textbf{Loop MS}, O'Beirne R, \& Safford MM (2017). Barriers to beta-blocker use and up-titration among patients with heart failure with reduced ejection fraction. \emph{Cardiovascular Drugs and Therapy} 31 (5-6), 559-564.}

\item \href{https://academic.oup.com/ibdjournal/article/24/3/641/4863703}{Sauer CG, \textbf{Loop MS}, Venkatexwaran S, Tangpricha V, Ziegler TR, Dhawan A, McCall C, Bonkowski E, Mack DR, Boyle B, Griffiths A, Leleiko NS, Keljo DJ, Markowitz J, Baker SS, Rosh J, Baldassano RN, Davis S, Patel S, Wang J, Marquis A, Spada KL, Kugathasan S, Walters T, Hyams JS, \& Denson LA (2017). Free and bioavailable 25-hydroxyvitamin D concentrations are associated with disease activity in pediatric patients with newly diagnosed treatment naive ulcerative colitis. \emph{Inflammatory Bowel Diseases} 24 (3), 641-650.}

\item \href{https://www.sciencedirect.com/science/article/pii/S1052305717301568}{McClure LA, \textbf{Loop MS}, Crosson W, Kleindorfer D, Kissela B, \& Al-Hamdan M (2017). Fine Particulate Matter (PM2.5) and the Risk of Stroke in the REGARDS Cohort. \emph{Journal of Stroke and Cerebrovascular Diseases} 26(8): 1739-1744.}

\item \href{http://circoutcomes.ahajournals.org/content/10/1/e003350.full?ijkey=Tq0l4EAFSjOGzX7\&keytype=ref}{\textbf{Loop MS}, Howard G, de los Campos G, Al-Hamdan MZ, Safford MM, Levitan EB, \& McClure LA (2017). Heat maps of hypertension, diabetes, and smoking in the continental US. \emph{Circulation: Cardiovascular Quality and Outcomes} \textbf{10}:e003350-00. doi: 10.1161/CIRCOUTCOMES.116.003350.}

\item \href{http://www.sciencedirect.com/science/article/pii/S0002914916304969}{\textbf{Loop MS}, Van Dyke MK, Chen L, Brown TM, Durant RW, Safford MM, and Levitan EB (2016). Comparison of Length of Stay, 30-Day Mortality, and 30-Day Readmission Rates in Medicare Patients With Heart Failure and With Reduced versus Preserved Ejection Fraction. \emph{American Journal of Cardiology} 118(1): 79 - 85.}

\item \href{http://www.ij-healthgeographics.com/content/14/1/4}{\textbf{Loop MS} \& McCLure LA (2015). Testing for clustering at many ranges inflates family-wise error rate (FWE). \emph{International Journal of Health Geographics} \textbf{14}(4): doi: 10.1186/1476-072X-14-4.}

\item \href{http://www.plosone.org/article/info\%3Adoi\%2F10.1371\%2Fjournal.pone.0075001#references}{\textbf{Loop MS}, Kent ST, Al-Hamdan MZ, Crosson WL, Estes SM, Estes, Jr. MG, Quattrochi DA, Hemmings SN, Wadley VG, \& McClure LA (2013). Fine particulate matter and incident cognitive impairment in the REasons for Geographic and Racial Differences in Stroke (REGARDS) cohort. \emph{PLoS ONE}. doi: 10.1371/journal.pone.0075001}

\item \href{http://www.frontiersin.org/Pharmacogenetics_and_Pharmacogenomics/10.3389/fgene.2012.00145/abstract}{\textbf{Loop MS}, Wood AC, Thomas AS, Dhurandhar EJ, Shikany JM, Gadbury GL, \& Allison DB (2012). Submitted for your consideration: potential advantages of a novel clinical trial design and initial patient reaction. \emph{Front. Gen.} \textbf{3}:145. doi: 10.3389/fgene.2012.00145.}
\end{etaremune}

%\section*{Papers in progress}
%\begin{enumerate}
%\item \textbf{Loop MS}, Van Dyke MK, Chen L, Brown TM, Durant RW, Safford MM, \& Levitan EB. Evidence-based beta blocker use associated with lower heart failure readmission and mortality, but not all-cause readmission, among Medicare beneficiaries hospitalized for heart failure with reduced ejection fraction. \emph{Under review}
%\item \textbf{Loop MS}, Lin F-C, Kim J, \& McClure LA. Increasing the power of the difference in Ripley's $K$ functions test in the presence of overlapping points. \emph{In preparation}
% Under review.
%\end{enumerate}

\section*{Invited talks}
\begin{itemize}
    \item ``Use and outcomes of beta blockers among Medicare beneficiaries hospitalized for heart failure with reduced ejection fraction.'' Pharmaceutical Outcomes and Policy, College of Pharmacy, University of Kentucky. October 13, 2021.
    \item ``Is CHD mortality still decreasing?: Findings from the ARIC study.'' Department of Epidemiology, University of Kentucky. December 6, 2018.
    \item ``Mapping risk factors in the wild.'' Department of Biostatistical Sciences, Wake Forest University. October 25, 2016.
\end{itemize}

\section*{Other oral Presentations}
\begin{itemize}
\item Lemas DJ (presenter), \textbf{Loop MS}, Duong M, Schleffer A, Collins C, Ciesielski A, Ridge ZD, Wagner J, \& Delcher C. ``Wastewater pharmacometabolomics: Feasibility study using liquid-chromatography mass spectrometry to estimate illicit drug consumption during college football games.'' Fall 2019 American Chemical Society National Meeting and Exposition.
	\item \textbf{Loop MS}, Van Dyke MK, Chen L, Safford MM, Kilgore ML, Brown TM, Durant RW, and Levitan EB. ``Potential contraindications and beta blocker prescription fill patterns among Medicare beneficiaries hospitalized with heart failure with reduced ejection fraction.'' 2016 UAB Postdoc Research Day. \textbf{2nd place in session}.
	\item \textbf{Loop MS} \& McClure LA. ``Trivial type 1 error rate inflation from testing for clustering at multiple ranges.'' Joint Statistical Meetings. August 3rd, 2014.
	\item \textbf{Loop MS}, McClure LA, Levitan EB, Al-Hamdan MZ, Crosson WL, \& Safford MM. ``PM$_{2.5}$ and Incident CHD: Are Hazard Ratios Enough?'' NHLBI Trainee Session of AHA EPI|NPAM 2014. March 19, 2014.
	\item \textbf{Loop MS}, de los Campos G, McClure LA, \& Howard G. ``Bayesian kriging of systolic blood pressure in the REGARDS cohort.'' UAB Section on Statistical Genetics Annual Retreat. April 19, 2013. 
	\item McCullough DJ \& \textbf{Loop MS}. ``Reproducible Research.'' UAB Biostatistics Journal Club. October 19, 2012.
	\item Wang G, Yan Q, \& \textbf{Loop MS}. ``Adaptive Designs.'' UAB Biostatistics Journal Club. March 14, 2012.
	\item \textbf{Loop MS}, Wood AC, Thomas AS, Shikany JM, Gadbury GL, \& Allison DB. ``Patient Interest in a Novel Design for Estimating Treatment Response Heterogeneity.'' Joint Statistical Meetings. August 4th, 2011.
	\item Lemas DJ \& \textbf{Loop MS}. ``Next-gen referencing: an introduction to Mendeley.'' UAB Section on Statistical Genetics Grant Writing Club. July 14th, 2011.
\end{itemize}

\section*{Posters}
\begin{itemize}
  \item \href{https://www.ahajournals.org/doi/abs/10.1161/circ.144.suppl_1.11676}{Saxon DT, Yang H, Li Q, \textbf{Loop MS}, Russell SD, Rossi J, Stacey R, \& Chang P. ``Do Heart Failure Admission Rates Change Around Holidays? The Atherosclerosis Risk in Communities (ARIC) Study.'' AHA Scientific Sessions 2021.} 
  \item \href{https://www.jacc.org/doi/pdf/10.1016/S0735-1097\%2821\%2904753-7}{Daubert MA, Stebbins A, Urrutia R, Chiswell K, \textbf{Loop M}, Harding C, Price T, Miller P, \& Wang T. ``Postpartum blood pressure screening in women with hypertensive disorders of pregnancy.'' ACC 2021.}
  
    \item DeBarmore BM, \textbf{Loop MS}, Astor BC, Matsushita K, Heiss GM, Rosamond WD, \& Franceschini N (2020). ``Incident Chronic Kidney Disease and Myocardial Infarction Without Chest Pain: Results from the Atherosclerosis Risk in Communities Study.'' AHA EPI|Lifestyle 2020.

    \item \textbf{Loop MS}, Fine JP, Folsom AR, Whitsel EA, Guild C, Ballantyne CM, \& Rosamond W. ``Secular Trends in Validity of Troponin I Assays for Myocardial Infarction Classification Among Four US Communities: Findings from the ARIC Study.'' AHA EPI|Lifestyle 2018.
    \item Ponce SG, Allison MA, Carlson JA, Perreira KM, \textbf{Loop MS}, Gonzalez II F, Isasi CR, Hurwitz BE, Penedo FJ, Kansai MM, Daviglus ML, Talavera GA, Rodriguez CJ, \& Gallo LG. ``Neighborhood Level Hispanic/Latino Ethnic Density and Left Ventricular Structure: ECHO-SOL and SOL-CASAS Ancillary Studies of Hispanic Community Health Study/Study of Latinos (HCHS/SOL).'' AHA EPI|Lifestyle 2018.
    \item Pasquel FJ, \textbf{Loop MS}, Menke A, O'Brien MJ, Vidot DC, Sotres-Alvarez D, Corsino L, Talavera GA, Gallo LC, Kaplan RC, Schneiderman N, Daviglus ML, Cowie CC, \& Aviles-Santa L. ``Progression From Prediabetes to Diabetes in Hispanics/Latinos. Results From the Hispanic Community Health Study/Study of Latinos (HCHS/SOL).'' AHA EPI|Lifestyle 2018.
    \item Kucharska-Newton A, Bullo M, \textbf{Loop MS}, Moore C, Haas SW, Rosamond W, Futreel W, Luo K, Yadav H, \& Bogle BM. ``Completeness in the Abstraction of Cardiac Biomarkers and Cardiac Pain Data From Electronic Health Records (EHR). Findings From the Atherosclerosis Risk in Communities (ARIC) Study.'' AHA EPI|Lifestyle 2018.
    \item Baldassari AR, Soliman EZ, Whitsel EA, Folsom AR, Matsushita K, Butler K, Bogle BM, Zhang Z-M, \textbf{Loop MS}, \& Heiss G. ``Feasibility of Electrocardiographic Identification and Classification of Myocardial Infarction Using Electronic Health Records. The Atherosclerosis Risk in Communities Study (ARIC).'' AHA EPI|Lifestyle 2018.
    \item Pena MB, Bulka CM, Swett K, Perreira K, Kansal M, \textbf{Loop MS}, Daviglus M, \& Rodriguez C. ``Occupational Environmental Exposures and Cardiac Structure and Function: The Echocardiographic Study of Latinos (ECHO-SOL).'' American College of Cardiology Conference 2018.
    \item Ricardo AC, \textbf{Loop MS}, Cedillo-Couvert EA, Chen Jinsong, Flessner MF, Franceschini N, Gonzalez II F, Kramer HJ, Moncrieft AE, Talavera GA, Daviglus ML, \& Lash JP. "Incidence of Chronic Kidney Disease (CKD) and Association of Major Cardiovascular Risk Factors with CKD in the Hispanic Community Health Study / Study of Latinos (HCHS/SOL)." AHA EPI|Lifestyle 2017.
	\item \textbf{Loop MS}, Van Dyke MK, Chen L, Brown TM, Durant RW, Safford MM, and Levitan EB. ``The association of beta blocker prescription fills with mortality and hospitalizations among Medicare beneficiaires hospitalized for heart failure with reduce ejection fraction.'' UAB Public Health Research Day 2016.
	\item \textbf{Loop MS}, Howard G, de los Campos G, Al-Hamdan MZ, Safford MM, Levitan EB, and McClure LA. ``Geographic patterns of the prevalence of major cardiovascular risk factors.'' AHA EPI|Lifestyle 2016.
	\item \textbf{Loop MS}, Van Dyke MK, Chen L, Brown TM, Durant RW, Safford MM, and Levitan EB. ``Length of stay and 30-day readmission rates similar between Medicare beneficiaries with different types of heart failure.'' AHA EPI|Lifestyle 2016.
	\item \textbf{Loop MS}, Van Dyke MK, Chen L, Brown TM, Durant RW, Safford MM, \& Levitan EB. ``Is there a mismatch between length of hospital stay and prognosis in Medicare beneficiaries hospitalized for heart failure?'' 2015 UAB Vascular Biology and Hypertension Symposium. September 24th, 2015.
	\item \textbf{Loop MS}, Howard G, Al-Hamdan MZ, and McClure LA. “Risk for hypertension varies across continental US.” UAB Comprehensive Cardiovascular Center (CCVC) 4th Annual Retreat. October 3, 2014.
	\item \textbf{Loop MS}, McClure LA, Levitan EB, Al-Hamdan MZ, Crosson WL, \& Safford MM. ``PM$_{2.5}$ and Incident CHD: Are Hazard Ratios Enough?'' UAB School of Public Health Sixth Annual Public Health Research Day. April 10th, 2014.
	\item McClure LA, \textbf{Loop MS}, Crosson WL, Kleindorfer DO, Kissela BM, \& Al-Hamdan MZ. ``Fine particulate matter (PM2.5) and the risk of stroke in the REGARDS cohort.'' International Stroke Conference. February, 2013. 
	\item \textbf{Loop MS}, Kent ST, Al-Hamdan MZ, Crosson WL, Estes SM, Estes MG, Quattrochi DA, Hemmings SN, Wadley VG, \& McClure LA. ``Air particulate matter and cognitive decline: a prospective cohort study.'' UAB School of Public Health Fourth Annual Public Health Research Day. April 2nd, 2012.
	\item \textbf{Loop MS}, Gadbury GL, Kaiser KA, Affuso O, \& Allison DB. ``How Large is Treatment Response Heterogeneity in Pharmaceutical Weight Loss Trials?: A Meta-analysis.'' UAB 		School of Public Health Third Annual Public Health Research Day. April 5th, 2011. (3rd place, Student Category)
	\item \textbf{Loop MS}, Wood AC, Thomas A, Shikany JM, Gadbury GL, \& Allison DB. ``Subject Opinion of a Novel Clinical Trial Design: A Survey.'' UAB School of Public Health Third Annual Public Health Research Day. April 5th, 2011.
	\item \textbf{Loop MS} \& Allison DB. ``Novel Design and Analysis for Estimating Treatment Response Heterogeneity.'' Section on Statistical Genetics Retreat. August, 2010.
	\item Buchanan ML, Eberly B, Eby B, \textbf{Loop MS}, Younanian M, Smith WH, \& Rissler LJ. 	``Responses of the Green Anole (Anolis carolinensis) to Fire-Restored Longleaf Forests.'' Undergraduate Research and Creative Activity Conference. The University of 		Alabama. April, 2009.
\end{itemize}

\section*{Teaching}

\begin{itemize}
  \item DPET 831 - Quantitative methods in clinical research (Spring 2021)
    \item BIOS 841 - Principles of statistical consulting (Spring 2019, 2020)
\end{itemize}

\section*{Students}

\begin{itemize}
    \item Lin Li - MS in 2020 (advisor)
    \item Joseph Kim - BSPH in 2020 (advisor)
    \item Betsy Hensel - MS in 2020 (reader)
    \item Haily Ausin - MS in 2020 (reader)
    \item Celia Stafford - MPH in 2020 (reader)
    \item Drew Thompson - MS in 2020 (reader)
    \item Griffin Bell - BSPH in 2018, MS in 2019 (advisor)
    \item Yunhan Wu - BSPH in 2019 (reader)
    \item Jimin Kim - MS in 2019 (reader)
\end{itemize}

\section*{Current funding}
\begin{itemize}

    \item \textbf{Interindividual variability in drug metabolism in ethnically diverse populations \\ 1R35GM143044-01, National Institute of General Medical Sciences 2021-07-01 to 2026-06-30 \\ Jackson (PI) \\ Role: Co-Investigator} \\ The goal of this proposal is to improve our understanding of how genetic and non-genetic factors affect drug metabolism and drug response in patients from understudied ethnic backgrounds. The proposed projects will investigate population-specific CYP genetic variants and novel phenotypic biomarkers to accurately quantify individual drug metabolism capacity in understudied ethnic populations.

    \item \textbf{Arterial stiffness and brain health in African Americans \\ R01AG062488, National Institute of Aging 2019-07-01 to 2023-04-30 \\
    Meyer (PI)\\
    Role: Co-Investigator}\\
    This study is focused on the stiffening of the arteries and the associated structural damage to small vessels in the brain and reduced cognitive function that precede cognitive impairment and Alzheimer's disease and related dementias. We propose to add a measure of artery stiffness to the 2020-2022 re-examination of the Jackson Heart Study; information on risk factors, damage to small vessels in the brain and cognitive performance will come from the existing study and as well as measures of artery stiffness in subset of participants that also participated in the 2011-2013 visit of the Atherosclerosis Risk in Communities- Neurocognitive study. This study will provide new information on how stiffening of the arteries relate to brain structure and function in African Americans that can inform efforts to reduce health disparities and to identify potential targets to lower the risk of cognitive impairment and Alzheimer's disease and related dementias.\\
    
    \item \textbf{Cell-based platform for gene delivery to the brain \\ R01NS102412, National Institute of Neurological Disorders and Stroke 2018-03-01 to 2022-11-30 \\ Batrakova (PI) \\ Role: Co-Investigator} \\ These studies will provide fundamental insights into how PBM interact with brain cells and facilitate horizontal gene transfer upon neurodegeneration, potentially opening up other cell-based gene delivery systems to the CNS and beyond.
    
    \item \textbf{Extracellular vesicles for CNS delivery of therapeutic enzymes to treat lysosomal storage disorders \\ R01NS112019 2019-09-01 to 2024-06-30 \\ Batrakova (PI) \\ Role: Co-Investigator} \\ Batten disease is a class of Lysosomal Storage Diseases (LSDs) that affects primarily children, who suffer without effective treatment options of enzyme replacement therapy (ERT). Our approach exploits the extracellular vesicles (EVs) released by specialized cells of the immune system, macrophages, as a platform for efficient and targeted brain delivery of therapeutic lysosomal enzymes. Successful development of this treatment modality would not only change the lives of LSDs patients, but could also lead to similar therapies for other diseases of the CNS.
    
    \item \textbf{Well-being initiative for women faculty of color to promote professional advancement in pharmacy and pharmaceutical sciences research \\ Amgen, Inc 2021-01-01 to 2022-12-31 \\ White (PI) \\ Role: Co-Investigator}
    
\end{itemize}

\section*{Previous funding}


\begin{itemize}
\item \textbf{Estimating patterns in the geographic variation and the social determinants of health that impact breastfeeding outcomes using natural language processing and electronic health records \\ UF Clinical and Translational Research Institute Grant 2020-07-01 to 2021-06-30\\ 
    Lemas (PI) \\
    Role: Co-Investigator} \\
The objective of this study is to leverage mom-baby linked EHR and biomedical informatics to estimate geospatial patterns in breastfeeding and characterize the SDoH that impact breastfeeding outcomes in vulnerable and hard-to-reach populations.

  \item \textbf{Optimizing Postpartum Cardiovascular Care in Women with Hypertensive Disorders of Pregnancy \\ Duke Clinical and Translational Science Institute 2019-07-01-2020-12-31\\  Daubert (Co-PI), Urrutia (Co-PI)\\ 
    Role: Co-Investigator}
    
\item \textbf{Adolescent Medicine Trials Network for HIV/AIDS Interventions (ATN) Coordinating Center Supplement\\1 U24 HD089880-01S1, E. K. Shriver National Institute of Child Health and Human Dev 2016-09-30 to present\\
LaVange (PI)\\
Role: Co-Investigator\\}
The goal of this U24 application is to establish a Coordinating Center for the next phase of the Adolescent Medicine Trials Network for HIV/AIDS Interventions (ATN) sponsored by the NIH. The objective of ATN is to increase awareness among at-risk youth of their HIV status and, for those with HIV, achieve linkage, engagement and retention in the care continuum.\\

\item \textbf{PeRiodontal Treatment to Eliminate Minority InEquality and Rural Disparities in Stroke \\ 16-2910 POUSC01-2000, National Institute on Minority Health and Health Disparities 2015-07-01 to 2020-06-30\\ Sen (PI)\\
    Role: PI of Data Coordinating Center}\\
    UNC will be a clinical site (Neurology and Dentistry), and provide entire project coordination (CSCC) for this phase III randomized controlled trial to test whether intensive periodontal treatment reduces the risk of recurrent vascular events amongst ischemic stroke and TIA survivors.\\
    
    \item \textbf{Atherosclerosis Risk in Communities (ARIC) Study - Coordinating Center\\HHSN268201700001I, National Heart, Lung and Blood Institute 2016-11-15 to 2021-11-14\\
Couper (PI)\\
Role: Co-Investigator}\\
The Atherosclerosis Risk in Communities Study (ARIC), sponsored by the NHLBI, is a prospective epidemiologic study conducted in four U.S. communities. ARIC is designed to investigate the etiology and natural history of atherosclerosis, the etiology of clinical atherosclerotic diseases, and variation in cardiovascular risk factors, medical care and disease by race, gender, location, and date.\\

    \item \textbf{VWF Biology and Disease Associations\\ Puget Sound Blood Center Research Institute 2015-06-01 to 2020-05-31\\ 
    Dong (PI)\\
    Role: Co-Investigator}\\
The studies proposed in this application are designed to improve our understanding of the role in sickle cell disease of a large,
sticky plasma protein called von Willebrand factor (VWF).\\

    \item \textbf{Epidemiologic Determinants of Cardiac Structure and Function Among Hispanics (ECHO-SOL2)\\ 2 R56 HL104199-05, National Heart, Lung and Blood Institute 2015-09-14 to 2018-05-31\\
    Rodriguez (PI)\\
    Role: PI of Data Coordinating Center}\\
    The proposed research represents a novel approach to understanding how progression o fHF risk factors (from pre-DM to DM, pre-HTN to HTN or from normal BMI to obesity) relate to changes in cardiac phenotypes and potentially to clinical HF.\\
    
    \item \textbf{Hispanic Community Health Study / Study of Latinos (HCHS/SOL): Coordinating Center\\HHSN2682013000011, National Heart, Lung and Blood Institute 2013-06-01 to 2019-05-31\\
Cai (PI)\\
Role: Co-Investigator}\\
This is an application for the University of North Carolina at Chapel Hill (UNC) to serve as the coordinating center for the Hispanic Community Health Study (HCHS). The HCHS study objectives are to identify the prevalence of and risk factors for diseases, disorders, and conditions in Hispanic populations and to determine the role of acculturation and disparities in the prevalence and development of these conditions. The HCHS coordinating center will provide operational and scientific coordination for the field centers to ensure that the research objectives of the study are met.\\

        \item \textbf{Predicting Response to Standard Pediatric Colitis Therapy: The PROTECT Study\\1-U01 DK095745-01, National Institute of Diabetes and Digestive and Kidney Diseases 2012-07-01 to 2017-07-31\\
Hyams (PI)\\
Role: Co-Investigator\\}
We hypothesize that a combination of clinical, genetic, and immunologic tests performed at diagnosis may allow construction of a model for individualized treatment and thereby improvement of current Ulerative Colitis outcomes.\\

	\item Geographic Distribution of Hypertension in U.S. Adults Age 45 or Older. \textbf{Competitive fellowship}, funded by the American Heart Association. 2014 Predoctoral Fellowship. Period: 1/1/2014 - 3/31/2015.\\
	
	\item Pre-doctoral Trainee, NHLBI T32HL079888, August 2012 - December 2013\\
	
	\item UAB Graduate School Fellow, August 2011 - July 2012\\
	
	\item Pre-doctoral Trainee, UAB Pre-doctoral Training Program in Obesity Related Research (1T32HL105349), December 2010 - July 2011\\
	
	\item Student Trainee, Kraft Fellowship Program of the UAB Nutrition and Obesity Research Center, June 2010 - November 2010
\end{itemize}

\section*{Honors and awards}
\begin{itemize}
    \item 2021 "Worthy of Recognition" as Instructor in PHCY 726: Research and Scholarship in Pharmacy III, UNC Eshelman School of Pharmacy
    \item 2018 Fellow of the American Heart Association
    \item 2018 Junior Faculty Development Award (\$10,000), UNC Chapel Hill
    \item 2015 Fellow, American Heart Association 10-Day Seminar on Epidemiology and Prevention of Cardiovascular Disease
	\item 2015 Career Enhancement Award, UAB Office of Postdoctoral Education 
	\item 2015 Outstanding PhD Student of the Year Award, UAB School of Public Health
\end{itemize}

\section*{Professional Activities}

\begin{itemize}
\item Reviewed papers for: \emph{JAMA Network Open (statistical reviewer)}, \emph{Journal of the American Heart Association}, \emph{Obesity}, \emph{PLoS ONE}, \emph{American Journal of Public Health}, \emph{American Journal of Clinical Nutrition}, \emph{Neurology}, \emph{Journal of Hospital Medicine}
\item Member, American Statistical Association, 2010 - present.
\item Member, IBS, ENAR, 2010 - present.
\item Member, American Heart Association, 2013 - present
\end{itemize}

\section*{Service}
\begin{itemize}
    \item Member, American Heart Association EPI Statistics Committee, 2017 - 2019, 2020 - 2022
    \item Abstract reviewer for: AHA EPI | Lifestyle 2019, 2020
	\item Judge for Graduate Biomedical Science Poster Session, UAB, November 2015, May 2016
\end{itemize}

\section*{Computing}
\begin{itemize}
    \item R, \LaTeX, git, SAS, UNIX/Linux, WinBUGS, High Performance Computing (HPC)
\end{itemize}

\section*{Languages}
\begin{itemize}
	\item English, fluent
	\item Spanish, limited working proficiency
\end{itemize}
\medskip

% Footer
\begin{center}
  \begin{small}
    Last updated: \today
  \end{small}
\end{center}

\end{document}